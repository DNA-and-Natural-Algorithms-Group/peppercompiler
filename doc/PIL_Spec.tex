\documentclass{article}
\usepackage{fullpage}

\usepackage{color}
\usepackage[normalem]{ulem}  % needed for \sout{} and \cut{}. 
\newcommand{\note}[1]{\textcolor{red}{#1}}
\newcommand{\add}[1]{\textcolor{blue}{#1}}
\newcommand{\cut}[1]{\sout{#1}}
\newcommand{\comment}[1]{}

\parindent 0cm
\parskip 1em

% Define code environment
\newenvironment{code}
{\vspace{-0.1in}\par\begin{list}{}{
%\setlength{\rightmargin}{\leftmargin}
\setlength{\listparindent}{0pt}
\raggedright
\setlength{\itemsep}{0pt}
\setlength{\parsep}{0pt}
\normalfont\ttfamily}
 \item[]}
{\end{list}\vspace{-0.1in}}


\begin{document}

\title{PIL Specification}
\author{Shawn Ligocki, Chris Berlind, Joseph Schaeffer, Erik Winfree}
\date{January 15, 2010}
\maketitle

PIL (Pepper Intermediate Language) is a language used by the Pepper
compiler framework to describe a DNA design specification. \add{It is
derived from the .DES design specification language created by Joe
Zadeh, Brian Wolfe, Niles Pierce, et al for the NUPACK multiobjective designer} and
has been extended to include kinetic and other information about DNA
designs. A PIL design contains information about every sequence
(domain), strand, structure (multi-stranded complex) and kinetic
reaction in a design.  \add{It is meant to provide a common framework for
sequence designers, including those used for pseudoknotted structures
such as DNA tiles.}




\section{Specification}

A specification is a list of statements. Each statement ends in a
line-break (newline character) but contains no line-breaks, so each non-blank line corresponds to exactly one statement.


\subsection{General lexical conventions}
\begin{itemize}
\item Integers are nonempty sequences of digits (regex: [0-9]+).

\item Floating point numbers will have the standard syntax used by C, Python, etc.\ (e.g.\ .1, -24.551, 1e6, 4.2E12, 1.1e-05, etc.).

\item Names are nonempty sequences of characters which are alphanumeric, underscore (\_), \textbf{or hyphen (-)}. The first character must \textbf{not} be a digit (regex: [a-zA-Z\_-][a-zA-Z0-9\_-]*). Names are identifiers for different objects in the specification, thus each object must have a unique name. Names \textbf{are} case sensitive.
\end{itemize}


\subsection{Comments and Whitespace}

Extraneous whitespace will be ignored. That is, wherever a sequence of at least one whitespace character (i.e.\ space or tab, but not newline) appears in the specification, the sequence will be treated as a single space character.

Everything on one line following a \texttt{\#} is a comment and will be ignored. The comment ends at the end of the line. There are no multiline comments.

Any line which contains only whitespace and/or a comment is considered blank and is ignored.

\pagebreak
Examples of comments:
\begin{code}
\# This is a comment

\#\#\# This is a comment too

sequence toe\_x = NNNNNN~:~6   \# This comment could talk about the sequence 
\end{code}

\subsection{Sequences}

A sequence is an indivisible segment of a nucleotide sequence. They are often called domains. Each sequence must be declared with a name, a length, and constraints on the allowable nucleotides.

Syntax:
\begin{code}
sequence <Name> = <Constraints>~:~<Length>
\end{code}
\texttt{<Constraints>} is a sequence of characters describing what base may be assigned to each nucleotide. Each character is one of ACGTRYWSMKBDHVN. The most common are N for any base, S for a strong base (C or G), W for a weak base (A or T), and A, T, C or G for a specific base. For a complete list see Table \ref{tab:Nucleotide-symbols}. Whitespace is allowed between symbols. 

\begin{table}[b]
\noindent \begin{centering}
\begin{tabular}{|c|c|c|c|c|c|c|c|c|c|c|c|c|c|c|c|}
\hline
Symbol & A & C & G & T & R & Y & W & S & M & K & B & D & H & V & N\\
\hline 
Allowed bases & A & C & G & T & AG & CT & AT & CG & AC & GT & CGT & AGT & ACT & ACG & ACGT\\
\hline
\end{tabular}
\par\end{centering}

\caption{\label{tab:Nucleotide-symbols}Nucleotide symbols}
\end{table}

\texttt{<Length>} is the integer length of the sequence. While explicitly including the length is redundant (as it can be inferred from the constraints), it is required as an error-checking step and to improve ease of readability.

Example sequence statements:
\begin{code}
sequence toe\_x = NNNNNN~:~6

sequence regulator = TCGGACT~:~7

sequence \_\_Anon-435 = SWABBBBBSS~:~10

sequence Translate-And-Gate-data\_x = SNNNNNNNNNNNNNNNNNNS~:~20
\end{code}

\subsection{Supersequences}

A supersequence is a named ordered collection of sequences (or other supersequences) grouped together for convenience. Like sequences, each supersequence must be declared with a length matching the sum of the lengths of its constituent sequences.

Syntax:
\begin{code}
supersequence <Name> = <List of (sequence | supersequence)>~:~<Length>
\end{code}
The \texttt{<List of (sequence | supersequence)>} is a space separated list. Each element in the list is the name of a sequence or supersequence optionally followed by an asterisk (\texttt{*}) meaning the Watson-Crick complement of the named sequence (\texttt{toe\_x*} means complement of sequence \texttt{toe\_x}).

While it would be possible to allow such groups to be declared using sequence statements, this would lead to ambiguity whenever a sequence is given a name consisting entirely of characters allowable for nucleotide constraints. For example, the string \texttt{TAG} could be used as a name or as sequence constraints, and in certain contexts, these two usages would be indistinguishable. To avoid this problem, the supersequence statement was introduced.

Examples of supersequence statements:
\begin{code}
supersequence x = toe\_x data\_x~:~21

supersequence input = x y{*} x~:~60
\end{code}


\add{Needs discussion of {\bf equal} statement.  Does this apply to sequences, supersequences, and/or strands?}

\subsection{Strands}

A strand statement represents an individual nucleic acid strand and
its sequence constraints. Strand statements follow a syntactic pattern
similar to supersequence statements, but while a supersequence is a
logical grouping of sequences, a strand represents a physical molecule
of DNA or RNA.

Syntax:
\begin{code}
strand <Name> = <List of (sequence | supersequence)>~:~<Length>
\end{code}
or \add{(as a new feature, not yet fully supported)} 
\begin{code}
strand [<Substrate type>] <Name> = <List of (sequence | supersequence)>~:~<Length>
\end{code}

As defined for supersequences, \texttt{<List of (sequence |
  supersequence)>} is a space-separated list of sequence or
supersequence names. \texttt{<Substrate type>} can be either
\texttt{DNA} or \texttt{RNA} to indicate the nucleic acid type of the
strand.

Examples of strand statements:
\begin{code}
strand X = x~:~21

strand Base = x{*} data\_y data\_z{*}~:~51
\end{code}

\subsection{Structures}

A structure statement represents a single- or multi-stranded ordered complex and its exact secondary structure (pseudoknotted or unpseudoknotted) at the base-pair level.

\add{NEEDS DESCRIPTION of domain-level specification and how strands' domains are calculated when supersequence definitions are involved.}

Syntax:
\begin{code}
structure <Name> = <List of strands>~:~<Secondary structure>
\end{code}
or
\begin{code}
structure [<Parameters>] <Name> = <List of strands>~:~<Secondary structure>
\end{code}
The \texttt{<List of strands>} is a list of strand names separated by plus signs (\texttt{+}). A strand \textbf{may} appear twice in a single structure.

\texttt{<Secondary structure>} is specified using dot-bracket
notation. To allow for pseudoknotted structures, \add{future versions will support} any bracket type (i.e.\ \texttt{(} \texttt{)}, \texttt{[} \texttt{]}, \texttt{\{} \texttt{\}}, \texttt{<} \texttt{>}) or labeled brackets (i.e.\ \texttt{(0} \texttt{0)}, \texttt{(1} \texttt{1)}, etc.) may represent paired bases, a dot (\texttt{.}) represents unpaired bases, and a plus sign represents a strand break. Whitespace is allowed between symbols \add{and must appear after number-labeled left brackets and before number-labeled right brackets}. 

\texttt{<Parameters>} uses the following syntax:
\begin{code}
<List of targets> \textsf{@} <List of conditions>
\end{code}
where \texttt{<List of targets>} and \texttt{<List of conditions>} are
comma-separated lists. \texttt{<List of conditions>} should contain a
set of conditions under which the set of targets in \texttt{<List of
targets>} should be satisfied. Possible target elements include
\texttt{no-opt} (meaning no optimization) and \texttt{<Float>nt}
(meaning within \texttt{<Float>} nucleotides of the target
structure, \add{e.g. on average}), \add{and \texttt{<Low> kcal/mol < dG < <High> kcal/mol} (meaning the free energy must be within the range).
\texttt{<Low>} and \texttt{<High>} may be \texttt{-inf} or \texttt{inf} to indicate no bounds are given. }
Possible condition elements include
\texttt{temperature=<Float>C} (temperature in degrees Celsius) and
\texttt{Na=<Float>M} (sodium ion concentration) \add{and
\texttt{Mg=<Float>M} (magnesium ion concentration)}.
\add{Units \texttt{K} for Kelvin, and \texttt{mM}, \texttt{uM}, \texttt{nM} are also supported.}


Examples of structure statements:
\begin{code}
structure Gate = Out + Base~:~(((((((((((...............+)))))))))))......

structure [no-opt] In\_Waste = In + Base~:~((((((((((((((+...))))))))))))))

structure [1nt \textsf{@} temperature=37.5C, Na=1.0M] IN = In~:~...................

structure pknot = s1 + s2~:~((((((((....[[[[[[[[....))))))))+....]]]]]]]]....
\end{code}

\subsection{Kinetics}

Kinetic statements express the reactions you want to take place between
structures. Kinetic reactions must preserve strands, thus each strand
in an input structure must appear in an output structure. Also, a
strand may not appear more than once in a kinetic reaction.

Syntax:
\begin{code}
kinetic <List of inputs> -> <List of outputs>
\end{code}
or
\begin{code}
kinetic [<Parameters>] <List of inputs> -> <List of outputs>
\end{code}
\texttt{<List of inputs>} and \texttt{<List of outputs>} are both lists of structure
names separated by plus sign (\texttt{+}). \texttt{<Parameters>} is currently being
used to specify target reaction rates. Its syntax is:
\begin{code}
<Low> /M/s < k < <High> /M/s

<Low> /s < k < <High> /s
\end{code}
where the units clearly depend on whether this is a bimolecular or
unimolecular reaction. \texttt{<Low>} and \texttt{<High>} are floating point numbers
and \texttt{<High>} may be \texttt{inf} for no upper bound.

For example:
\begin{code}
kinetic IN + Gate -> OUT + IN\_Waste

kinetic [0 /M/s < k < 1e6 /M/s] Left + SeesawRight -> Right + SeesawLeft

kinetic [1000.1 /s < k < inf /s] Glob -> X + Y
\end{code}

\subsection{Noninteracting}

Noninteracting statements are used to specify specific negative design
targets. These are sets of things that we do not want to interact
with each other.

Generally, the syntax is:
\begin{code}
noninteracting [<Type>] <List of things>
\end{code}
Where \texttt{<List of things>} could be sequences, strands or structures and
their interpretation depends up on the \texttt{<Type>}. Currently, we have
one possible type:
\begin{code}
noninteracting [kinetic] <List of structures>
\end{code}
where \texttt{<List of structures>} is a space separated list of structures
which should pairwise not react kinetically (and not waste too much
time interacting). It is specifically used by Chris Berlind's KinD.

\add{Additional noninteracting lists will be used by Spurious_Design:  \texttt{sticky-ends} for tile sticky ends, \texttt{toeholds} for strand displacement toeholds, \texttt{branch-migration} for 3-way and 4-way branch migration domains, and \texttt{single-stranded} for single-strandedsequences that must not hybridize to each other.}



\subsection{Related software tools}

\add{Software that currently (November 2010) reads or writes PIL include: 
  \texttt{spurious_design.py} (the wrapper for the \texttt{spuriousC} sequence-symmetry designer),
  \texttt{KinD.py} (strand displacement kinetics network evaluator and sequence designer),
  \texttt{random_design.py} (a trivial random sequence designer),
  \texttt{PIL-to-DES.py} (an yet-to-be-written converter for using the NUPACK designer).}

\end{document}
