\documentclass{article}
\usepackage{fullpage}

\parindent 0cm
\parskip 1em

% Define code environment
\newenvironment{code}
{\par\begin{list}{}{
\setlength{\rightmargin}{\leftmargin}
\setlength{\listparindent}{0pt}
\raggedright
\setlength{\itemsep}{0pt}
\setlength{\parsep}{0pt}
\normalfont\ttfamily}
 \item[]}
{\end{list}}


\begin{document}

\title{PIL Specification}
\author{Shawn Ligocki}
\date{January 15, 2010}
\maketitle

PIL (Pepper Intermediate Language) is a language used by the Pepper
compiler framework to describe a DNA design specification. It is loosely
based on the .des design specification language created by Joe Zadeh
for the NUPACK designer and has been extended to include kinetic and
other information about DNA designs. A PIL design contains information
about every sequence (domain), strand, structure (multi-stranded complex)
and kinetic reaction in a design.


\section{Specification}

A specification is a list of statements. Each statement ends in a
line-break (but contains no line-breaks) and so each non-blank line
in a specification corresponds to exactly one statement.


\subsection{General lexical conventions}
\begin{itemize}
\item Integers are non-empty sequences of digits (regex: [0-9]+).

\item Floating point numbers will have the standard syntax used by C, Python, etc.\ (e.g.\ .1, -24.551, 1e6, 4.2E12, 1.1e-05, etc.).

\item Names are non-empty sequences of characters which are alphanumeric,
underscore (\_) \textbf{or hyphen (-)}. The first character must \textbf{not}
be a digit (regex: [a-zA-Z\_-][a-zA-Z0-9\_-]*). Names are
identifiers for different objects in the specification, thus each
object must have a unique name. Names \textbf{are}\textbf{\emph{ }}case
sensitive.
\end{itemize}


\subsection{Comments and Whitespace}

Whenever there is a whitespace in the specification, any number of
space or tab characters may be used and will be ignored.

Everything on one line following a \texttt{\#} is a comment and will
be ignored. The comment ends at the end of the line. There are no
multiline comments.

Any line which contains only whitespace and/or a comment is ignored.
\begin{code}
\#~This~is~a~comment

\#\#\#~This~is~a~comment~too

sequence~toe\_x~=~6N~:~6~~~\#~This~comment~could~talk~about~the~sequence~
\end{code}

\subsection{Sequences}

Sequences are the indivisible segments of a nucleotide sequence. They
are often called domains. You must name each sequence and explicitly
enter the length and any constraints on the nucleotides that may be
used.

Syntax:
\begin{code}
sequence~<Name>~=~<Constraints>~:~<Length>
\end{code}
\texttt{<Constraints>} is a sequence of characters describing what base may
be assigned to each nucleotide. Each is one of ACGTRYWSMKBDHVN. The
most common are N for any base, S for a strong base (C or G), W for
a weak base (A or T) and A, T, C or G for a specific base. For a complete
list see Table \ref{tab:Nucleotide-symbols}.

\begin{table}[b]
\noindent \begin{centering}
\begin{tabular}{|c|c|c|c|c|c|c|c|c|c|c|c|c|c|c|c|}
\hline 
Symbol & A & C & G & T & R & Y & W & S & M & K & B & D & H & V & N\\
\hline 
Allowed bases & A & C & G & T & AG & CT & AT & CG & AC & GT & CGT & AGT & ACT & ACG & ACGT\\
\hline
\end{tabular}
\par\end{centering}

\caption{\label{tab:Nucleotide-symbols}Nucleotide symbols}
\end{table}

Whitespace is allowed between symbols. \texttt{<Length>} is the integer length
of the sequence.

For example:
\begin{code}
sequence~toe\_x~=~NNNNNN~:~6

sequence~regulator~=~TCGGACT~:~7

sequence~\_\_Anon-435~=~SWABBBBBSS~:~10

sequence~Translate-And-Gate-data\_x~=~SNNNNNNNNNNNNNNNNNNS~:~20
\end{code}

\subsection{Supersequences}

Supersequences are collections of sequences (or other supersequences)
grouped together for convenience.

Syntax:
\begin{code}
supersequence~<Name>~=~<List~of~(sequence~|~supersequence)>~:~<Length>
\end{code}
The \texttt{<List~of~(sequence~|~supersequence)>} is a space separated list. Each element
in the list is the name of a sequence or supersequence optionally
followed with an asterisk (\texttt{*}) meaning the Watson-Crick complement
of the named sequence (\texttt{toe\_x*} means complement of sequence \texttt{toe\_x}).

For example:
\begin{code}
supersequence~x~=~toe\_x~data\_x~:~21

supersequence~input~=~x~y{*}~x~:~60
\end{code}

\subsection{Strands}

Strands identify the actual sequence constraints that a strand will
have. But, syntactically, they are identical to supersequences.

Syntax:
\begin{code}
strand~<Name>~=~<List~of~sequences>~:~<Length>
\end{code}
Examples:
\begin{code}
strand~X~=~x~:~21

strand~Base~=~x{*}~data\_y~data\_z{*}~:~51
\end{code}

\subsection{Structures}

Structures express the DNA complexes you want including a precise
target secondary structure.

Syntax:
\begin{code}
structure~<Name>~=~<List~of~strands>~:~<Secondary~structure>
\end{code}
or
\begin{code}
structure~{[}<Parameters>{]}~<Name>~=~<List~of~strands>~:~<Secondary~structure>
\end{code}
The \texttt{<List of strands>} is a list of strand names separated by plus
signs (\texttt{+}). You may not use the same strand more than once in a structure;
if you want to have two identical strands in a structure, you must
define two identical strands. \texttt{<Secondary structure>} is in dot-paren
notation using plus sign as the strand break symbol. Whitespace is
allowed between symbols. \texttt{<Parameters>} is currently being used only
to specify the optimization goal. It may either be no-opt (for no
optimization) or \texttt{<Integer>nt} (for within \texttt{<Integer>} nucleotides of the target structure).

For example:
\begin{code}
structure~Gate~=~Out~+~Base~:~((((((((((((((((....................+))))))))))))))))......

structure~{[}no-opt{]}~IN\_Waste~=~In~+~Base~:~(((((((((((((((((((+...)))))))))))))))))))

structure~{[}1nt{]}~IN~=~In~:~...................
\end{code}

\subsection{Kinetics}

Kinetic statements express the reactions you want to take place between
structures. Kinetic reactions must preserve strands, thus each strand
in an input structure must appear in an output structure. Also, a
strand may not appear more than once in a kinetic reaction.

Syntax:
\begin{code}
kinetic~<List~of~inputs>~->~<List~of~outputs>
\end{code}
or
\begin{code}
kinetic~{[}<Parameters>{]}~<List~of~inputs>~->~<List~of~outputs>
\end{code}
\texttt{<List of inputs>} and \texttt{<List of outputs>} are both lists of structure
names separated by plus sign (\texttt{+}). \texttt{<Parameters>} is currently being
used to specify target reaction rates. Its syntax is:
\begin{code}
<Low>~/M/s~<~k~<~<High>~/M/s

<Low>~/s~<~k~<~<High>~/s
\end{code}
where the units clearly depend on whether this is a bimolecular or
unimolecular reaction. \texttt{<Low>} and \texttt{<High>} are floating point numbers
and \texttt{<High>} may be \texttt{inf} for no upper bound.

For example:
\begin{code}
kinetic~IN~+~Gate~->~OUT~+~IN\_Waste

kinetic~{[}0~/M/s~<~k~<~1e6~/M/s{]}~Left~+~SeesawRight~->~Right~+~SeesawLeft

kinetic~{[}1000.1~/s~<~k~<~inf~/s{]}~Glob~->~X~+~Y
\end{code}

\subsection{Noninteracting}

Noninteracting statements are used to specify specific negative design
targets. These are sets of things that we do not want to interact
with each other.

Generally, the syntax is:
\begin{code}
noninteracting~{[}<Type>{]}~<List~of~things>
\end{code}
Where \texttt{<List of things>} could be sequences, strands or structures and
their interpretation depends up on the \texttt{<Type>}. Currently, we have
one possible type:
\begin{code}
noninteracting~{[}kinetic{]}~<List~of~structures>
\end{code}
where \texttt{<List of structures>} is a space separated list of structures
which should pairwise not react kinetically (and not waste too much
time interacting). It is specifically used by Chris Berlind's KinD.
\end{document}
